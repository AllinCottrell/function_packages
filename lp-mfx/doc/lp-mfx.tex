\documentclass{article}
\usepackage{doc,url,verbatim,fancyvrb}
\usepackage{pifont,amsmath}
\usepackage{pkgheads}
\usepackage[letterpaper,body={6.3in,9.15in},top=.8in,left=1.1in]{geometry}
\usepackage[pdftex,hyperfootnotes=false]{hyperref}

\usepackage[utf8]{inputenc}
\DeclareUnicodeCharacter{2212}{-}

% \usepackage[a4paper,body={6.1in,9.7in},top=.8in,left=1.1in]{geometry}

\begin{document}

\setlength{\parindent}{0pt}
\setlength{\parskip}{1ex}
\setcounter{secnumdepth}{2}

\newcommand{\argname}[1]{\textsl{#1}}
  
\title{The \textsf{lp-mfx} package, version 1.0}
\author{Allin Cottrell}
\date{April, 2019}
\maketitle

\section{Introduction}
\label{sec:intro}

This package computes marginal effects for a certain set of limited
dependent variable estimators, namely binary logit and probit, ordered
logit and probit, multinominal logit and logistic regression. See the
gretl commands \texttt{logit}, \texttt{probit} and \texttt{logistic}
for details on these. It can be called from the \textsf{Analysis} menu
in a GUI window displaying a supported model (see
section~\ref{sec:gui}), or via scripting (see
section~\ref{sec:script-usage}).

The effects computed are derivatives with respect to the regressors
(or the outcome of a discrete change from 0 to 1 in the case of 0/1
dummy regressors). For binary logit and probit the numerator or target
of these derivatives is the estimated probability that the dependent
variable equals 1; in the ordered and multinomial cases it is the
estimated probability of each discrete outcome in turn; and for
logistic regression it is the predicted value of the untransformed
dependent variable. In most cases (details below) standard errors for
these effects are available, computed via the delta method, along with
$z$-scores and $P$-values for the null hypothesis of no effect.

\section{Modes of marginal effects}
\label{sec:modes}

Three modes are supported as follows:
\begin{enumerate}
\item Marginal effects at the means of the regressors.
\item Average marginal effects.
\item Marginal effects at each observation in the sample range.
\end{enumerate}
The first of these modes is fast, since only one calculation is
required for each regressor/outcome pair. Average marginal effects are
more complex since a calculation must be performed for each
observation used in estimation. If the number of observations is large
this can take a while; the process can be speeded up by choosing not
to produce standard errors for the effects.

Per-observation marginal effects are available only when the dependent
variable is ``unitary'' (either binary or logistic). Standard errors
are not produced in this case.

\section{Primary and non-primary regressors}

Besides detecting, and giving special treatment to, 0/1 dummies among
the regressors, \textsf{lp-mfx} attempts to identify two common cases
of ``non-primary'' regressors, namely
\begin{itemize}
\item Regressor $x_j$ is the square of regressor $x_i$ (quadratic
  specification).
\item Regressor $x_k$ is the product of regressors $x_i$ and $x_j$
  (interaction term).
\end{itemize}
If the model specification is, to take a simple binary probit example,
\[
  P(y=1\, |\, x, \beta) = \Phi(\beta_0 + \beta_1 x_1 + \beta_2 x_1^2 +
  \beta_3 x_2 + \beta_4 x_3 + \beta_5 x_2 x_3) = \Phi(x\beta)
\]
then the marginal effects for the primary regressors $x_1$, $x_2$ and
$x_3$ are
\begin{align*}
  \partial  P/\partial x_1 &=\phi(x\beta)\cdot (\beta_1 + 2 \beta_2x_1) \\
  \partial  P/\partial x_2 &=\phi(x\beta)\cdot (\beta_3 + \beta_5x_3) \\
  \partial  P/\partial x_3 &=\phi(x\beta)\cdot (\beta_4 + \beta_5x_2)
\end{align*}
while the non-primary terms, $x_1^2$ and $x_2 x_3$, do not possess
marginal effects of their own.

To ensure that such non-primary terms are detected correctly, it is
advisable to create them within gretl; any of the following
commands will give the series in question labels that make them
identifiable:
\begin{verbatim}
series x1sq = x1*x1
series x1p2 = x^2
square x1 # produces sq_x1
series interact = x2*x3
series iact = x2 * x3
\end{verbatim}
The names given to the series don't matter, only the expressions used
to generate them; and in the expressions spaces or their absence don't
matter.

If you are using a third-party dataset that includes non-primary terms
that are not identified as such by descriptive labels, as an
alternative to recreating the relevant series you may use the utility
function \texttt{lp\_mfx\_fixlist()} supplied by \textsf{lp-mfx}.
This takes a named list argument.  Each series in the list is checked,
by brute-force comparison of data values, and if it is the square of
another series in the list, or the product of any other two series,
this fact is recorded in its label. As an example of usage, suppose we
are using as regressors series \texttt{x1} to \texttt{x5}; then we
might do
\begin{verbatim}
list X = x1 x2 x3 x4 x5
lp_mfx_fixlist(X)
probit y const X
\end{verbatim}
after which we can be confident that \textsf{lp-mfx} will deal
correctly with any squares or interactions among the regressors in
\texttt{X}.

Note that more elaborate cases of non-primary regressors are not
handled automatically at present (for example, cubes or higher powers
of a regressor, or interaction terms that are the products of three or
more variables).

\section{Scripting basics}
\label{sec:script-usage}

After issuing the command \texttt{include lp-mfx.gfn} the following
primary function is available:
\begin{verbatim}
bundle lp_mfx (const bundle mod,
               int mode[1:3:1],
               bool skip_se[0])
\end{verbatim}
The \texttt{mod} argument (the only required one) can be obtained
after estimation of a suitable model via the
\verb|$model| accessor, as in
\begin{verbatim}
logit y 0 x1 x2 x3
bundle mod = $model
\end{verbatim}
The optional second argument---an integer value 1, 2 or 3---can be
used to select the mode of the marginal effects, as set out in
section~\ref{sec:modes}. The default is 1, giving effects at the
means.

There's also a variant of this function which allows you to supply a
specific vector of values of the independent variables at which
marginal effects should be evaluated:
\begin{verbatim}
bundle lp_mfx_atx (const bundle mod,
                   const matrix x,
                   bool skip_se[0])
\end{verbatim}
The argument \texttt{x} must be a row vector of length equal to that
of the model's parameter vector (including the constant, if present).

The optional third argument to \texttt{lp\_mfx} and
\texttt{lp\_mfx\_atx} can be used to suppress the computation of
standard errors.

The return value from these functions is a bundle containing
(primarily) one or more matrices holding the marginal effects. The
results can be displayed in tabular form via the function
\texttt{lp\_mfx\_print}, which takes as its single argument a
``pointer to'' the bundle:
\begin{verbatim}
bundle mf = lp_mfx($model)
lp_mfx_print(&mf)
\end{verbatim}


\section{GUI access}
\label{sec:gui}

On installing the \texttt{lp\_mfx} package you should get the choice
of letting it attach to the \textsf{Analysis} menu in gretl model
windows under the item \textsf{Marginal effects}. In that case when you
estimate a suitable model via the graphical interface you can view the
marginal effects by clicking on this item.

\section{Examples}
\label{sec:examples}

We confine ourselves below to datasets supplied with gretl, so it
should be possible to follow along if you wish.  Here's a simple
binary probit example, using the \texttt{mroz87} dataset which is to
do with women's labor force participation:

{\small
\begin{verbatim}
include lp-mfx.gfn
open mroz87.gdt
probit LFP 0 KL6 K618 WA WE
bundle b = lp_mfx($model, 1)
lp_mfx_print(&b)
bundle b = lp_mfx($model, 2)
lp_mfx_print(&b)
\end{verbatim}
}
which produces (partial output):
{\small
\begin{verbatim}
Binary probit marginal effects at means
LFP = 1, Pr = 0.5733

            dp/dx        s.e.           z        pval        xbar
 KL6     -0.34753    0.044200     -7.8627  3.7585e-15     0.23772
K618    -0.021842    0.015719     -1.3895     0.16468      1.3533
  WA    -0.015009   0.0029216     -5.1370  2.7910e-07      42.538
  WE     0.047075   0.0086964      5.4132  6.1910e-08      12.287

Binary probit average marginal effects

            dp/dx        s.e.           z        pval
 KL6     -0.31238    0.034676     -9.0084  2.0914e-19
K618    -0.019633    0.014082     -1.3941     0.16328
  WA    -0.013490   0.0024938     -5.4097  6.3138e-08
  WE     0.042314   0.0073701      5.7412  9.3981e-09
\end{verbatim}
}

In this case we see that the marginal effects at the regressor means
and the average marginal effects are not very different. Both show
that women are less likely to participate in the labor force
(\texttt{LFP}) when they have children under the age of 6
(\texttt{KL6}), and when they are older (\texttt{WA}), but more likely
to participate when their education level (\texttt{WE}) is higher. The
example is simple in that all the regressors are ``primary'' and none
is a dummy variable.

As a variant on the above, we might ask what the marginal effects look
like for a 30-year old woman with 15 years of education and no
children. The additional command is

{\small
\begin{verbatim}
matrix x = {1, 0, 0, 30, 15}
bundle b = lp_mfx_atx($model, x)
lp_mfx_print(&b)
\end{verbatim}
}
and the output is
{\small
\begin{verbatim}
Binary probit marginal effects at specified x
LFP = 1, Pr = 0.8991

            dp/dx        s.e.           z        pval        xval
 KL6     -0.15658    0.026928     -5.8148  6.0712e-09      0.0000
K618   -0.0098410   0.0061266     -1.6063     0.10821      0.0000
  WA   -0.0067621  0.00084402     -8.0118  1.1300e-15      30.000
  WE     0.021210   0.0047280      4.4860  7.2563e-06      15.000
\end{verbatim}
}

\vspace{1ex}

A second example uses the \texttt{keane} dataset. The dependent
variable, \texttt{status}, has three unordered values: 1 indicates
that the respondent (a young man) is in school, 2 that he is at home,
and 3 that he is in work. The \textsf{lp-mfx} output shows marginal
effects on the probability of each outcome but here we focus on the
``in school'' outcome. The script

{\small
\begin{verbatim}
include lp-mfx.gfn
open keane.gdt -q
smpl (year==87) --restrict
logit status 0 educ exper expersq black --multinomial
bundle b = lp_mfx($model)
lp_mfx_print(&b)
bundle b = lp_mfx($model, 2)
lp_mfx_print(&b)
\end{verbatim}
}
produces (in part)
{\small
\begin{verbatim}
Multinomial logit marginal effects at means
note: dp/dx based on discrete change for black

Outcome 1: (status = 1, Pr = 0.0212)
             dp/dx        s.e.           z        pval        xbar
 educ    0.0073312   0.0015143      4.8412  1.2907e-06      12.549
exper   -0.0054125   0.0019287     -2.8063   0.0050119      3.4403
black   -0.0073531   0.0055632     -1.3217     0.18625     0.37973

Multinomial logit average marginal effects
note: dp/dx based on discrete change for black 

Outcome 1: (status = 1)
             dp/dx        s.e.           z        pval
 educ     0.017379   0.0029057      5.9808  2.2201e-09
exper    -0.019438   0.0042472     -4.5766  4.7269e-06
black    -0.017991    0.011646     -1.5449     0.12238
\end{verbatim}
}
In this case the average marginal effects are larger and more sharply
estimated than the effects at the means. Note that the regressor
\texttt{expersq}, the square of \texttt{exper} (years of work
experience), is correctly excluded from the list of marginal effects:
the effect shown for \texttt{exper} takes account of the quadratic
specification.

Anyone interested in a comparison with \textsf{Stata} (and in
possession of a copy of said software) can append the following to the
basic gretl script
{\small
\begin{verbatim}
foreign language=stata --send-data
  quietly mlogit status educ exper c.exper#c.exper i.black
  margins, dydx(educ exper black) atmeans predict(outcome(1))
  margins, dydx(educ exper black) predict(outcome(1))
end foreign
\end{verbatim}
} to reveal that the answers are the same as those shown above. Note
that special syntax is required on the \texttt{mlogit} command line to
get the square of \texttt{exper} recognized as such, and to get
\texttt{black} treated as a 0/1 dummy rather than as continuous. Also
the \texttt{margins} command produces average marginal effects by
default but the \texttt{atmeans} option can be used to get the effects
at the means.


\end{document}
