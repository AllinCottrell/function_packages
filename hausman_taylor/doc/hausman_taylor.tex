\documentclass{article}
\usepackage{verbatim,fancyvrb}
\usepackage[authoryear]{natbib}
\usepackage[pdftex]{graphicx}
\usepackage{pict2e}
\usepackage{gretl}
\usepackage[letterpaper,body={6.3in,9.15in},top=.8in,left=1.1in]{geometry}
\usepackage{color,hyperref}

% \usepackage[a4paper,body={6.1in,9.7in},top=.8in,left=1.1in]{geometry}

\definecolor{steel}{rgb}{0.03,0.20,0.45}

\hypersetup{pdftitle={hausman\_taylor version 1.0},
            pdfauthor={Allin Cottrell},
            colorlinks=true,
            linkcolor=blue,
            urlcolor=red,
            citecolor=steel,
            bookmarksnumbered=true,
            plainpages=false
}

\begin{document}

\setlength{\parindent}{0pt}
\setlength{\parskip}{1ex}

\newcommand{\argname}[1]{\textsl{#1}}

\title{hausman\_taylor version 1.0}
\author{Allin Cottrell}
%\date{December 12, 2025}
\maketitle

\section{Introduction}

This package estimates a panel-data model using the method of
\cite{hausman-taylor81}. As is well known,
%
\begin{itemize}
\item the standard fixed-effects estimator cannot handle
  time-invariant variables (since nothing remains of such variables
  after sweeping out the individual means), while
\item the random-effects estimator cannot handle (on pain of
  inconsistency) regressors that are correlated with the unobserved
  individual effects.\footnote{See the chapter titled ``Panel data'' in
    the \textit{Gretl User's Guide} for an extended discussion of these
    points.}
\end{itemize}

The Hausman--Taylor estimator bridges this gap: it permits estimation
of a model that includes both time-invariant terms and regressors that
are correlated with the individual effects. The ``price of admission''
(more formally, the condition for identification) is that there must
be at least as many time-varying exogenous regressors---``exogenous''
in the sense of being uncorrelated with the individual effects---as 
there are time-invariant regressors that are suspected of endogeneity,
that is, of being correlated with the individual effects.

\section{The model}
\label{sec:model}

Let $i=1,\dots,N$ index individuals and $t=1,\dots,T$ index time. The
model is
\begin{equation}
\label{eq:htmod}
y_{it} = \beta_0 + x_{1it}'\beta_1 + x_{2it}'\beta_2 +
 z_{1i}'\gamma_1 + z_{2i}'\gamma_2 + u_i + \epsilon_{it}
\end{equation}
where $x_1$ and $x_2$ are time-varying and $z_1$ and $z_2$ are
time-invariant. The variables $x_1$ and $z_1$ are exogenous
(uncorrelated with the individual effects, $u_i$) while $x_2$ and
$z_2$ are assumed to be so correlated. All of the regressors are
assumed to be uncorrelated with $\epsilon_{it}$.

In general, $x_{1it}$, $x_{2it}$, $z_{1i}$ and $z_{2i}$ are vectors
of length $k_1$, $k_2$, $g_1$ and $g_2$, respectively, subject to the
identification requirement $k_1 \geq g_2$.

The algorithm for the Hausman--Taylor estimator---for a balanced panel
in which the time-series length, $T$, is the same for all
individuals---is commonly given as follows:

\begin{enumerate}
\item Regress $\tilde{y} = (y_{it} - \bar{y}_i)$ on
  $\tilde{x}_1 = (x_{1it} - \bar{x}_{1i})$ and
  $\tilde{x}_2 = (x_{2it} - \bar{x}_{2i})$ to obtain initial estimates
  of $\beta_1$ and $\beta_2$. Use the residuals from this
  fixed-effects regression, $e_{it}$, to estimate the ``within'' error
  variance $\sigma^2_{\epsilon}$.
\item Perform an IV regression of the stacked individual means of
  $e_{it}$ on $z_1$ and $z_2$, using as instruments $z_1$ and $x_1$.
  Use the residual variance from this regression, $s_2^2$, to
  estimate $\sigma^2_u$ as $s_2^2 - \hat{\sigma}^2_{\epsilon}/T$, and
  calculate the GLS coefficient
\begin{equation}
\label{eq:theta}
\theta = 1 - \left(\frac{\hat{\sigma}^2_{\epsilon}}
  {\hat{\sigma}^2_{\epsilon} + T\hat{\sigma}^2_u}\right)^{0.5}
\end{equation}
\item Let $w_{it} \equiv (x_{1it}, x_{2it}, z_{1i}, z_{2i})$.  Run
  an IV regression of $y_{it}^* = (y_{it} - \theta\bar{y}_i)$ on
  $w_{it}^* = (w_{it} - \theta\bar{w}_{i})$, using as instruments
  $\tilde{x}_1$, $\tilde{x}_2$, $\bar{x}_{1}$ and $z_{1}$.
\end{enumerate}

The final step can also be described thus: regress the quasi-demeaned
dependent variable on the quasi-demeaned regressors, taking as
instruments the fully-demeaned time-varying regressors, the individual
means of the exogenous time-varying terms, and the levels of the
exogenous time-invariant terms. As \citet[p.\ 1393]{hausman-taylor81}
remark, ``Making use of time-varying variables in two ways---to
estimate their own coefficients and to serve as instruments for
endogenous time-invariant variables---allows identification and
efficient estimation of both $\beta$ and $\gamma$.''

In an unbalanced panel the time-series length, $T_i$, differs across
individuals. In that case steps 2 and 3 above have to be modified
slightly. First, the calculation of $\hat{\sigma}^2_u$ uses the harmonic
mean of the $T_i$s in place of a common $T$. Second, the value of
$\theta$ differs across individuals:
%
\begin{equation}
\label{eq:theta-unbal}
\theta_i = 1 - \left(\frac{\hat{\sigma}^2_{\epsilon}}
  {\hat{\sigma}^2_{\epsilon} + T_i\hat{\sigma}^2_u}\right)^{0.5}
\end{equation}
%
Section~\ref{sec:unbal} below takes up a further issue pertaining to
the unbalanced case.

\section{The \dtk{hausman_taylor} function}

The signature of this function is
%
\begin{code}
bundle hausman_taylor (series y "dependent variable", 
                       list Lexo "exogenous regressors",
                       list Lndo "endogenous regressors", 
                       int verbosity[0:2:1],
                       bool as_stata[0])
\end{code}

The series \texttt{y} is the dependent variable, and the lists
\texttt{Lexo} and \texttt{Lndo} correspond, respectively, to $x_1$ plus
$z_1$ and $x_2$ plus $z_2$ in equation~(\ref{eq:htmod}).  This
function does not undertake to judge which regressors are exogenous
and which endogenous---you must decide the partition between
\texttt{Lexo} and \texttt{Lndo}---but it can easily determine which
regressors are time-varying and which invariant.

The \texttt{verbosity} parameter accepts values 0, 1 or 2: a value of
0 means that nothing is printed; 1 means that the Hausman--Taylor
estimates are printed; 2 means that in addition the results of the
preliminary regressions (described in Section~\ref{sec:model}) are
printed. The default value is 1.

The \dtk{as_stata} boolean can be used to produce results
comparable with Stata's \textsf{xthtaylor} command; an explanation of
this option is given on page~\pageref{stata2}.\label{stata1}

This function returns a bundle containing the items shown in
Table~\ref{tab:bun}. Most of the contents should be fairly
self-explanatory, but the following comments may be useful.
\begin{itemize}
\item In the case of an unbalanced panel, the GLS coefficients
  $\theta_i$ will differ across individuals; the \texttt{theta} value
  is then the mean of the $\theta_i$s.
\item The \texttt{Wald}, \texttt{Htest} and \texttt{Stest} matrices,
  if present, are each row vectors containing test statistic, degrees
  of freedom and $P$-value pertaining to the Wald, Hausman and Sargan
  tests, respectively.  The Wald test uses the coefficient vector and
  covariance matrix to test the null hypothesis that only the constant
  truly has a non-zero coefficient. The Hausman and Sargan tests are
  available only if the specification is overidentified ($k_1 >
  g_2$). They both test the null hypothesis of correct specification.
  The Hausman test is based on a vector of contrasts, the difference
  between the fixed-effects and Hausman--Taylor estimates of the
  coefficients on the time-varying regressors. The Sargan test is
  based on the explained sum of squares from a regression of the
  Hausman--Taylor residuals on all of the instruments. Small
  $P$-values on these tests cast doubt on the consistency of the
  estimator.
\end{itemize}

\begin{table}[htbp]
\centering
\begin{tabular}{llp{.6\textwidth}}
  \textit{name}   & \textit{type} & \textit{description} \\[4pt]
  \texttt{depvar} & string & name of the dependent variable \\
  \texttt{parnames} & strings & names of regressors \\
  \texttt{Lexo} & list & exogenous regressors \\
  \texttt{Lndo} & list & endogenous regressors \\
  \texttt{coeff} & matrix & regression coefficients \\
  \texttt{stderr} & matrix & standard errors \\
  \texttt{vcv} & matrix & variance--covariance matrix \\
  \dtk{s_e} & scalar & square root of ``within'' variance, 
    $\hat{\sigma}^2_{\epsilon}$ \\
  \dtk{s_u} & scalar & square root of $\hat{\sigma}^2_{u}$ \\
  \texttt{ncoeff} & scalar & total number of coefficients \\
  \texttt{nobs} & scalar & total number of observations used \\
  \texttt{effn} & scalar & number of units included \\
  \texttt{Tmin} & scalar & minimum $T_i$ value \\
  \texttt{Tmax} & scalar & maximum $T_i$ value \\
  \texttt{theta} & scalar & GLS coefficient, $\theta$ \\
  \texttt{Wald} & matrix & Wald test results \\
  \texttt{Htest} & matrix & Hausman test results \\
  \texttt{Stest} & matrix & Sargan test results \\
  \texttt{yhat} & series & fitted values \\
  \texttt{uhat} & series & residuals \\
  \texttt{rsq} & scalar & corr$(y,\hat{y})^2$ \\
\end{tabular}
\caption{Items in \dtk{hausman_taylor} bundle}
\label{tab:bun}
\end{table}

\section{Sample script}

The sample script for this package uses data on (log) wages and
several covariates for 595 individuals observed annually from 1976 to
1982, taken from the US Panel Study of Income Dynamics. These data
were originally employed by \cite{cornwell88} to assess various
instrumental-variable estimators for panel data including
Hausman--Taylor. They  were revisited by \cite{baltagi-akom90} and in
chapter 7 of \cite{baltagi05}. The script replicates both sets of
estimates; partial output is shown in Listing~\ref{sample-output}.

\begin{script}[htbp]
  \label{sample-output}
Cornwell and Rupert specification

\begin{outbit}
Hausman-Taylor estimates for lwage
using 4165 observations (n = 595, T = 7)

             coefficient    std. error       z      p-value 
  ----------------------------------------------------------
  const       2.88442       0.852777       3.382    0.0007   ***
  wks         0.000909009   0.000598818    1.518    0.1290  
  south       0.00713766    0.0325480      0.2193   0.8264  
  smsa       -0.0417623     0.0194019     -2.152    0.0314   **
  ms         -0.0363440     0.0188575     -1.927    0.0539   *
  exper       0.112972      0.00246967    45.74     0.0000   ***
  exper2     -0.000419119   5.45872e-05   -7.678    1.62e-14 ***
  occ        -0.0213946     0.0137801     -1.553    0.1205  
  ind         0.0188416     0.0154404      1.220    0.2224  
  union       0.0303548     0.0148964      2.038    0.0416   **
  fem        -0.136847      0.127280      -1.075    0.2823  
  blk        -0.281829      0.176627      -1.596    0.1106  
  ed          0.140525      0.0658715      2.133    0.0329   **

  sigma_u = 0.94172543
  sigma_e = 0.15180272
  theta   = 0.93918626

Hausman test: chi-square(3) = 14.5555 [0.0022]
Sargan test:  chi-square(3) = 14.8759 [0.0019]
\end{outbit}

Baltagi's specification

\begin{outbit}
Hausman-Taylor estimates for lwage
using 4165 observations (n = 595, T = 7)

             coefficient    std. error       z      p-value 
  ----------------------------------------------------------
  const       2.91273       0.283652      10.27     9.76e-25 ***
  occ        -0.0207047     0.0137809     -1.502    0.1330  
  south       0.00743984    0.0319550      0.2328   0.8159  
  smsa       -0.0418334     0.0189581     -2.207    0.0273   **
  ind         0.0136039     0.0152374      0.8928   0.3720  
  exper       0.113133      0.00247095    45.79     0.0000   ***
  exper2     -0.000418865   5.45981e-05   -7.672    1.70e-14 ***
  wks         0.000837403   0.000599732    1.396    0.1626  
  ms         -0.0298507     0.0189800     -1.573    0.1158  
  union       0.0327714     0.0149084      2.198    0.0279   **
  fem        -0.130924      0.126659      -1.034    0.3013  
  blk        -0.285748      0.155702      -1.835    0.0665   *
  ed          0.137944      0.0212485      6.492    8.47e-11 ***

  sigma_u = 0.94180300
  sigma_e = 0.15180272
  theta   = 0.93919126

Hausman test: chi-square(3) = 5.25773 [0.1539]
Sargan test:  chi-square(3) = 5.22910 [0.1558]
\end{outbit}
\end{script}

In each case we may take it that the endogenous regressor of primary
interest is \texttt{ed} (education level). The specifications differ in
their treatment of two pairs of regressors: in Cornwell and Rupert
\texttt{wks} (weeks worked) and \texttt{ms} (marital status) are taken
to be exogenous while \texttt{occ} (blue-collar dummy) and \texttt{ind}
(manufacturing dummy) are endogenous; Baltagi reverses this, assuming
that \texttt{wks} and \texttt{ms} are endogenous, \texttt{occ} and and
\texttt{ind} exogenous. Baltagi finds a slightly smaller, but more
sharply estimated, return to education. The Hausman and Sargan
specification tests favor Baltagi's specification.

\section{Graphical interface}

An entry-point for \textsf{hausman\_taylor} can be found under the
\textsf{Panel} sub-menu of gretl's \textsf{Model} menu: the label is
\textsf{Hausman-Taylor}. The dialog is shown in Figure~\ref{fig:gui}.

\begin{figure}[htbp]
  \centering
  \includegraphics[scale=0.6]{ht_gui}
  \caption{Specify arguments for \dtk{hausman_taylor}}
  \label{fig:gui}
\end{figure}

\section{Ancillary printing function}

The ancillary public function \dtk{ht_print()} is provided to
``pretty-print'' the results contained in the bundle provided by
\dtk{hausman_taylor()}; \dtk{ht_print()} takes a pointer to the bundle
as its sole argument.

\section{More on the unbalanced case}
\label{sec:unbal}

A noteworthy aspect of the Hausman--Taylor estimator is the treatment
of $x_1^*$---that is, quasi-demeaned $x_1$---in the final IV
regression. If the panel is balanced $x_1^*$ is effectively treated as
exogenous. It does not appear explicitly among the instruments, but we
have the exact linear relationship
\[
x^*_{1it} \equiv (x_{1it} -\theta \bar{x}_{1i}) = 
  (x_{1it} - \bar{x}_{1i}) + (1-\theta) \bar{x}_{1i}
 = \tilde{x}_{1it} + (1-\theta) \bar{x}_{1i}
\]
so that $x_1^*$ is ``perfectly instrumented'' by $\tilde{x}_{1}$ and
$\bar{x}_1$. This is as it should be. By assumption $x_{1it}$ is
independent of $u_i$, and therefore so is $\bar{x}_{1i}$. The
transformation $x^*_{1it} = x_{1it} - \theta \bar{x}_{1i}$ clearly
does not introduce any dependence on $u_i$, so $x_1^*$ ought to be
treated as exogenous. It is not included as an instrument simply
because it would be redundant, given the point made above.

Now consider the unbalanced case. It is standard to calculate
$\sigma^2_u$ as $s_2^2 - \sigma^2_{\epsilon}/\bar{T}$, where
$\bar{T}$ is the harmonic mean of the $T_i$s. And $\theta$ varies by
individual according to (\ref{eq:theta-unbal}) above.  This means
there is no longer an exact linear relationship between $x_1^*$ and
the instruments $\tilde{x}_1$ and $\bar{x}_1$, which raises the
question, should $x_1^*$ be added to the set of instruments in the
final IV step of Hausman--Taylor?

The alternative---including $\tilde{x}_1$ and $\bar{x}_1$ as
instruments, but not $x_1^*$---amounts to treating $x_1^*$ as
endogenous, but there is no reason for this. When the panel is
unbalanced $x^*_{1it}$ is defined by
\[
x^*_{1it} = x_{1it} - \theta_i \bar{x}_{1i}
\]
The substitution of $\theta_i$ for the common $\theta$ in the balanced
case doesn't make any relevant difference to the status of $x_1^*$.
The only way in which individual-specific information enters
$\theta_i$ is via the number of observations, $T_i$, and there is no
reason to believe that $T_i$ should be correlated with $u_i$.

We conclude that failing to include $x_1^*$ as an instrument in the
unbalanced case will degrade the efficiency of the estimator. Yet this
is what is done in Stata's \textsf{xthtaylor} command and in R's
\textsf{plm} package.

If the argument above is correct, it should be possible to show via
simulation the degradation of the efficiency of Hausman--Taylor when
quasi-demeaned $x_1$ is treated as endogenous in the final IV
regression, given unbalanced data. Conversely, if the argument above
is wrong then presumably simulation should produce evidence of
inconsistency when quasi-demeaned $x_1$ is added as an instrument.

To explore this we ran a simulation of the following form. 
\begin{enumerate}
\item For $K = 5000$ iterations, generate a random dataset with a
  known set of parameter values and a correlation structure that
  respects the Hausman--Taylor assumptions. Randomly assign missing
  values to some proportion of the observations.
\item For each dataset, run the Hausman--Taylor procedure both ways
  (respectively omitting and including $x_1^*$ as an instrument in the
  final stage) and record the parameter estimates.
\item Calculate the mean and standard deviation of $\hat{\theta} -
  \theta$ for each parameter $\theta$.
\end{enumerate}

Specifically, we constructed datasets containing one variable in each
of the categories $x_1$, $x_2$, $z_1$ and $z_2$, using the parameter
values $\beta_0 = \beta_1 = \beta_2 = \gamma_1 = \gamma_2 =
1$. The panel comprised $T=10$ observations for each of $N$
individuals. The series were constructed as follows:
\begin{align*}
 u_{i} &= N(0,1) \\
 x_{1it} &= N(0,1) \\
 x_{2it} &= N(0,1) + au_i \\
 z_{1i} &= N(0,1) \\
 z_{2i} &= N(0,1) + au_i + b\bar{x}_{1i} \\
 \epsilon_{it} &= N(0,1) \\
\end{align*}
with $a = 0.3$ and $b=0.5$.  The formula for $x_{2it}$ ensures that
$x_2$ is endogenous; and that for $z_{2i}$ ensures both the
endogeneity of $z_2$ and its correlation with $x_1$, which is wanted
so that $x_1$ can serve as an instrument for $z_2$.
In the case of the time-invariant variables, random sequences of
length $N$ were generated and the value for each individual
was entered at all $T$ observations.

After constructing the data a uniform random series, $v$, was
generated on $[0,1)$ and the value of the dependent variable was set
to ``missing'' at observations for which $v_{it} < 0.04$, giving an
expectation of 4 percent unusable observations, hence unbalancing the
panel.

One further point: since $\sigma^2_u$ is estimated indirectly, in finite
samples it may happen that $\hat{\sigma}^2_u \le 0$, in which case the
standard procedure is to set $\theta = 0$. This erases the distinction
we're interested in, between the two variants of the Hausman--Taylor
estimator. It's therefore necessary to calibrate the simulation so that
a non-positive $\hat{\sigma}^2_u$ doesn't arise too often.\footnote{In
  addition we discarded iterations in which $\hat{\sigma}^2_u$ was
  non-positive, continuing until the specified number of replications
  was reached with non-zero $\theta$.}

Figure~\ref{fig:means} shows results for the mean error of estimation,
relevant to assessing consistency. Points are shown for four values of
$N$: 20, 50, 100 and 200. The results differ only marginally for
$\beta_1$ and $\beta_2$, while for $\beta_0$, $\gamma_1$ and $\gamma_2$
the results are better when $x_1^*$ included as an instrument. There's
no evidence of inconsistency in the latter case.

Figure~\ref{fig:sds} shows the standard deviation of estimate minus
parameter, relevant to assessing efficiency; results are again given
for four values of $N$.  The relative performance of the variants
strongly supports the contention made above. While performance with
respect to $\beta_1$ and $\beta_2$ is virtually identical, the
$\beta_0$ and $\gamma$ estimates show \textit{much} greater variance
when $x_1^*$ is not included as an instrument; indeed, the variance of
the $\gamma$ estimates is such that they may be useless in practice.

We are now in a position to explain the \dtk{as_stata} option for the
\dtk{hausman_taylor} function mentioned on
page~\pageref{stata1}.\label{stata2} This option has no effect for
balanced panels, but in the unbalanced case it means ``Do what Stata
does---that is omit $x_1^*$ as an instrument in the final
regression---even though we reckon it's not the right thing to do.''

As a practical point, it should be noted that the inclusion (in the
unbalanced case) of $x_1^*$ as an instrument alongside $\tilde{x}_1$
and $\bar{x}_1$ in the final Hausman--Taylor regression may produce
near-singularity of the instrument matrix, depending on the dataset
(this was evident in the simulations). However, with modern
econometric software this does not pose a serious problem, since
redundant instruments will be dropped automatically.


\begin{figure}[p]
  \centering
\input means.tex
\caption{Mean error of estimates ($y$-axis) against number of individuals,
$N$, in sample ($x$-axis).
{\color{red} Red}: $x_1^*$ excluded from the set of instruments;
{\color{blue} blue}, $x_1^*$ included.
$T = 10$ and 5000 replications.}
\label{fig:means}
\end{figure}

\begin{figure}[p]
  \centering
\input sds.tex
\caption{Standard error of estimates ($y$-axis) against number of individuals,
$N$, in sample ($x$-axis).
{\color{red} Red}: $x_1^*$ excluded from the set of instruments;
{\color{blue} blue} $x_1^*$ included.
$T = 10$ and 5000 replications.}
\label{fig:sds}
\end{figure}

% \clearpage

\bibliographystyle{gretl}
\bibliography{gretl}

\end{document}
