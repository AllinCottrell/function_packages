\documentclass{article}
\usepackage{doc,verbatim,fancyvrb}
%\usepackage{pifont}
\usepackage[authoryear]{natbib}
% \usepackage[pdftex]{graphicx}
\usepackage{pict2e}
\usepackage{gretl}
\usepackage[letterpaper,body={6.3in,9.15in},top=.8in,left=1.1in]{geometry}
\usepackage{color,hyperref}

% \usepackage[a4paper,body={6.1in,9.7in},top=.8in,left=1.1in]{geometry}

\definecolor{steel}{rgb}{0.03,0.20,0.45}

\hypersetup{pdftitle={hausman\_taylor version 0.3},
            pdfauthor={Allin Cottrell},
            colorlinks=true,
            linkcolor=blue,
            urlcolor=red,
            citecolor=steel,
            bookmarksnumbered=true,
            plainpages=false
}

\begin{document}

\setlength{\parindent}{0pt}
\setlength{\parskip}{1ex}

\newcommand{\argname}[1]{\textsl{#1}}

\title{hausman\_taylor version 0.3}
\author{Allin Cottrell}
\date{October 11, 2024}
\maketitle

\section{Introduction}

This package estimates a panel-data model using the method of
\cite{hausman-taylor81}. As is well known (see the chapter titled
``Panel data'' in the \textit{Gretl User's Guide} for an extended
discussion)
%
\begin{itemize}
\item the standard fixed-effects estimator cannot handle
  time-invariant variables (since nothing remains of such variables
  after sweeping out the individual means), while
\item the random-effects estimator cannot handle (on pain of
  inconsistency) regressors that are correlated with the unobserved
  individual effects.
\end{itemize}

The Hausman--Taylor estimator bridges this gap: it permits estimation
of a model that includes both time-invariant terms and regressors that
are correlated with the individual effects. The ``price of admission''
(more formally, the condition for identification) is that there must
be at least as many time-varying exogenous regressors---``exogenous''
in the sense of being uncorrelated with the individual effects---as 
there are time-invariant regressors that are suspected of endogeneity,
that is, of being correlated with the individual effects.

\section{The model}
\label{sec:model}

Let $i=1,\dots,N$ index individuals and $t=1,\dots,T$ index time. The
model is
\begin{equation}
\label{eq:htmod}
y_{it} = \beta_0 + x_{1it}'\beta_1 + x_{2it}'\beta_2 +
 z_{1i}'\alpha_1 + z_{2i}'\alpha_2 + u_i + \epsilon_{it}
\end{equation}
where $x_1$ and $x_2$ are time-varying and $z_1$ and $z_2$ are
time-invariant. The variables $x_1$ and $z_1$ are exogenous
(uncorrelated with the individual effects, $u_i$) while $x_2$ and
$z_2$ are assumed to be so correlated. All of the regressors are
assumed to be uncorrelated with $\epsilon_{it}$.

In general, $x_{1it}$, $x_{2it}$, $z_{1i}$ and $z_{2i}$ are vectors
of length $k_1$, $k_2$, $g_1$ and $g_2$, respectively, subject to the
identification requirement $k_1 \geq g_2$.

The algorithm for the Hausman--Taylor estimator---for a balanced panel
in which the time-series length, $T$, is the same for all
individuals---is commonly given as follows:

\begin{enumerate}
\item Regress $\tilde{y} = (y_{it} - \bar{y}_i)$ on
  $\tilde{x}_1 = (x_{1it} - \bar{x}_{1i})$ and
  $\tilde{x}_2 = (x_{2it} - \bar{x}_{2i})$ to obtain initial estimates
  of $\beta_1$ and $\beta_2$. Use the residuals from this regression,
  $e_{it}$, to estimate the ``within'' error variance
  $\sigma^2_{\epsilon}$.
\item Perform an IV regression of the stacked individual means of
  $e_{it}$ on $z_1$ and $z_2$, using as instruments $z_1$ and $x_1$.
\item Use the residual variance from step 2, namely $s_2^2$, to
  estimate $\sigma^2_u$ as $s_2^2 - \hat{\sigma}^2_{\epsilon}/T$, and
  calculate the GLS coefficient
\begin{equation}
\label{eq:theta}
\theta = 1 - \left(\frac{\hat{\sigma}^2_{\epsilon}}
  {\hat{\sigma}^2_{\epsilon} + T\hat{\sigma}^2_u}\right)^{0.5}
\end{equation}
\item Finally, perform an IV regression of $y_{it}^* = (y_{it} -
  \theta\bar{y}_i)$ on $w_{it}^* = (w_{it} - \theta\bar{w}_{i})$,
  where $w_{it} = (x_{1it}, x_{2it}, z_{1i}, z_{2i})$, using as
  instruments $\tilde{x}_1$, $\tilde{x}_2$, $\bar{x}_{1}$ and
  $z_{1}$.
\end{enumerate}

In an unbalanced panel the time-series length, $T_i$, differs across
individuals. In that case step 3 above has to be modified. First, the
calculuation of $\hat{\sigma}^2_u$ uses the harmonic mean of the
$T_i$s in place of a common $T$. Second, the value of $\theta$ differs
across individuals:
%
\begin{equation}
\label{eq:theta-unbal}
\theta_i = 1 - \left(\frac{\hat{\sigma}^2_{\epsilon}}
  {\hat{\sigma}^2_{\epsilon} + T_i\hat{\sigma}^2_u}\right)^{0.5}
\end{equation}
%
Section~\ref{sec:unbal} below takes up a further issue pertaining to
the unbalanced case.

\section{The hausman\_taylor function}

The signature of this function is
%
\begin{code}
bundle hausman_taylor (series y "dependent variable", 
                       list X1 "time-varying exogenous vars",
                       list X2 "time-varying endogenous vars", 
                       list Z1 "time-invariant exogenous vars",
                       list Z2 "time-invariant endogenous vars",
                       int verbosity[0:2:1],
                       bool emulate_stata[0])
\end{code}

The series \texttt{y} is the dependent variable, and the four list
arguments correspond the subsets of regressors $x_1$, $x_2$, $z_1$ and
$z_2$ as shown in equation~(\ref{eq:htmod}).

The \texttt{verbosity} parameter accepts values 0, 1 or 2: a value of
0 means that nothing is printed; 1 means that the Hausman--Taylor
estimates are printed; 2 means that in addition the results of the
preliminary regressions (described in Section~\ref{sec:model}) are
printed. The default value is 1.

The \texttt{emulate\_stata} boolean can be used to produce results
comparable with Stata's \textsf{xthtaylor} command; an explanation of
this option is given on page~\pageref{stata2}.\label{stata1}

By way of return value, this function offers a gretl bundle containing
the items shown in Table~\ref{tab:bun}. Most of the contents should be
fairly self-explanatory, but the following comments may be useful.
\begin{itemize}
\item In the case of an unbalanced panel, the GLS coefficients
  $\theta_i$ will differ across individuals; the \texttt{theta} value
  is then the mean of the $\theta_i$s.
\item The \texttt{Htest} and \texttt{Stest} matrices, if present, are
  row vectors containing test statistic, degrees of freedom and
  $P$-value pertaining to the Hausman or Sargan test, respectively.
\end{itemize}

\begin{table}[htbp]
\centering
\begin{tabular}{llp{.6\textwidth}}
  \textit{name}   & \textit{type} & \textit{description} \\[4pt]
  \texttt{yname} & string & name of the dependent variable \\
  \texttt{xnames} & string & names of regressors (space separated) \\
  \texttt{coeff} & matrix & regression coefficients \\
  \texttt{stderr} & matrix & standard errors \\
  \texttt{vcv} & matrix & variance--covariance matrix \\
  \texttt{s\_e} & scalar & square root of ``within'' variance, 
    $\hat{\sigma}^2_{\epsilon}$ \\
  \texttt{s\_u} & scalar & square root of $\hat{\sigma}^2_{u}$ \\
  \texttt{ninst} & scalar & number of instruments \\
  \texttt{nobs} & scalar & total number of obsevations used \\
  \texttt{effn} & scalar & number of units included \\
  \texttt{Tmin} & scalar & minimum $T_i$ value \\
  \texttt{Tmax} & scalar & maximum $T_i$ value \\
  \texttt{theta} & scalar & GLS coefficient, $\theta$ \\
  \texttt{Htest} & matrix & Hausman test results \\
  \texttt{Stest} & matrix & Sargan test results \\
\end{tabular}
\caption{Items in hausman\_taylor bundle}
\label{tab:bun}
\end{table}

\section{Sample script}

The sample script for this package estimates two models of (log) wages
for 595 individuals observed annually from 1976 to 1982.  The
data---taken from the US Panel Study of Income Dynamics---were
originally analysed by \cite{cornwell88} and employed for assessing
various instrumental-variable estimators for panel models (including
the Hausman--Taylor model). These data are also analysed by
\cite{baltagi-akom90} and in chapter 7 of \cite{baltagi05}. The two
models are the original specification from Cornwell and Rupert, and a
modified specification from Baltagi.

\section{Graphical interface}

An entry-point for \textsf{hausman\_taylor} can be found under the
\textsf{Panel} sub-menu of gretl's \textsf{Model} menu: the label is
\textsf{Hausman-Taylor}.

% See Figure~\ref{fig:gui}.
% \begin{figure}[htbp]
%   \centering
%   \includegraphics[scale=0.6]{ivpanel-gui}
%   \caption{Specify arguments for ivpanel}
%   \label{fig:gui}
% \end{figure}

\section{Auxiliary printing function}

The auxiliary public function \texttt{ht\_print()} is provided to
``pretty-print'' the results contained in the bundle provided by
\texttt{hausman\_taylor()}; \texttt{ht\_print()} takes a pointer to
the bundle as its sole argument.

\section{More on the unbalanced case}
\label{sec:unbal}

A noteworthy aspect of the Hausman--Taylor estimator is the treatment
of $x_1^*$---that is, quasi-demeaned $x_1$---in the final IV
regression. If the panel is balanced $x_1^*$ is effectively treated as
exogenous. It does not appear explicitly among the instruments, but we
have the exact linear relationship
\[
x^*_{1it} \equiv (x_{1it} -\theta \bar{x}_{1i}) = 
  (x_{1it} - \bar{x}_{1i}) + (1-\theta) \bar{x}_{1i}
 = \tilde{x}_{1it} + (1-\theta) \bar{x}_{1i}
\]
so that $x_1^*$ is ``perfectly instrumented'' by $\tilde{x}_{1}$ and
$\bar{x}_1$. This is as it should be. By assumption $x_{1it}$ is
independent of $u_i$, and therefore so is $\bar{x}_{1i}$. The
transformation $x^*_{1it} = x_{1it} - \theta \bar{x}_{1i}$ clearly
does not introduce any dependence on $u_i$. So $x_1^*$ ought to be
treated as exogenous. It is not included as an instrument simply
because it would be redundant, given the point made above.

Now consider the unbalanced case. It is standard to calculate
$\sigma^2_u$ as $\sigma^{*2} - \sigma^2_{\epsilon}/\bar{T}$, where
$\bar{T}$ is the harmonic mean of the $T_i$s. And $\theta$ varies by
individual according to (\ref{eq:theta-unbal}) above.  This means
there is no longer an exact linear relationship between $x_1^*$ and
the instruments $\tilde{x}_1$ and $\bar{x}_1$, which raises the
question, should $x_1^*$ be added to the set of instruments in the
final IV step of Hausman--Taylor?

The alternative---including $\tilde{x}_1$ and $\bar{x}_1$ as
instruments, but not $x_1^*$---amounts to treating $x_1^*$ as
endogenous, but there is no reason for this. When the panel is
unbalanced, $x^*_{1it}$ is defined by
\[
x^*_{1it} = x_{1it} - \theta_i \bar{x}_{1i}
\]
The substitution of $\theta_i$ for the common $\theta$ in the balanced
case doesn't make any relevant difference to the status of $x_1^*$.
The only way in which individual-specific information enters
$\theta_i$ is via the number of observations, $T_i$, and there is no
reason to believe that $T_i$ should be correlated with $u_i$.

I conclude that failing to include $x_1^*$ as an instrument in the
unbalanced case will degrade the efficiency of the estimator. Yet this
is what is done in Stata's \textsf{xthtaylor} command and in R's
\textsf{plm} package.

If the argument above is correct, it should be possible to show via
simulation the degradation of the efficiency of Hausman--Taylor when
quasi-demeaned $x_1$ is treated as endogenous in the final IV
regression, given unbalanced data. Conversely, if my argument
is wrong then presumably simulation should produce evidence of
inconsistency when quasi-demeaned $x_1$ is added as an instrument.

To explore this I ran a simulation of the following form. 
\begin{enumerate}
\item For $K = 5000$ iterations, generate a random dataset with a
  known set of parameter values and a correlation structure that
  respects the Hausman--Taylor assumptions. Randomly assign missing
  values to some proportion of the observations.
\item For each dataset, run the Hausman--Taylor procedure both ways
  (respectively omitting and including $x_1^*$ as an instrument in the
  final stage) and record the parameter estimates.
\item Calculate the mean and standard deviation of $\hat{\theta} -
  \theta$ for each parameter $\theta$.
\end{enumerate}

Specifically, I constructed datasets containing one variable in each
of the categories $x_1$, $x_2$, $z_1$ and $z_2$, using the parameter
values $\beta_0 = \beta_1 = \beta_2 = \alpha_1 = \alpha_2 =
1$. The panel comprised $T=10$ observations for each of $N$
individuals. The series were constructed as follows:
\begin{align*}
 u_{i} &= N(0,1) \\
 x_{1it} &= N(0,1) \\
 x_{2it} &= N(0,1) + au_i \\
 z_{1i} &= N(0,1) \\
 z_{2i} &= N(0,1) + au_i + b\bar{x}_{1i} \\
 \epsilon_{it} &= N(0,1) \\
\end{align*}
with $a = 0.3$ and $b=0.5$.  The formula for $x_{2it}$ ensures that
$x_2$ is endogenous; and that for $z_{2i}$ ensures both the
endogeneity of $z_2$ and its correlation with $x_1$, which is wanted
so that $x_1$ can serve as an instrument for $z_2$.
In the case of the time-invariant variables, random sequences of
length $N$ were generated and the value for each individual
was entered at all $T$ observations.

After constructing the data a uniform random series, $v$, was
generated on $[0,1)$ and the value of the dependent variable was set
to ``missing'' at observations for which $v_{it} < 0.04$, giving an
expectation of 4 percent unusable observations, hence unbalancing the
panel.

One further point should be noted. Since $\sigma^2_u$ is estimated
indirectly, it is quite possible in finite samples to get a zero or
negative value for $\hat{\sigma}^2_u$, in which case the standard
procedure is to set $\theta = 0$. This destroys the distinction I'm
interested in, between the two variants of the Hausman--Taylor
estimator. It's therefore necessary to calibrate the simulation so
that a non-positive $\hat{\sigma}^2_u$ doesn't arise too
often.\footnote{In addition I discarded iterations in which
  $\hat{\sigma}^2_u$ was non-positive, continuing until the specified
  number of replications was reached with non-zero $\theta$.}

Figure~\ref{fig:means} shows results for the mean error of
estimation, relevant to assessing consistency. Points are shown for
four values of $N$, 20, 50, 100 and 200. Note that the results differ
only marginally for $\beta_1$ and $\beta_2$, while for $\beta_0$,
$\alpha_1$ and $\alpha_2$ the results are better when $x_1^*$ included
as an instrument. There's no evidence of inconsistency in the latter
case.

Figure~\ref{fig:sds} shows the standard deviation of estimate minus
parameter, relevant to assessing efficiency; results are again given
for $N$ equal to 20, 50, 100 and 200.  Here we see that the relative
performance of the variants strongly supports the contention made
above. While performance with respect $\beta_1$ and $\beta_2$ is
virtually identical, the $\beta_0$ and $\alpha$ estimates show
\textit{much} greater variance when $x_1^*$ is not included as an
instrument; indeed, the variance of the $\alpha$ estimates is
such that they may be useless in practice.

We're now in a position to explain the \texttt{emulate\_stata} option
for the \texttt{hausman\_taylor} function mentioned on
page~\pageref{stata1}.\label{stata2} This option has no effect for
balanced panels, but in the unbalanced case it means ``Do what Stata
does---that is omit $x_1^*$ as an instrument in the final
regression---even though we reckon it's wrong.''

As a practical point, it should be noted that the inclusion (in the
unbalanced case) of $x_1^*$ as an instrument alongside $\tilde{x}_1$
and $\bar{x}_1$ in the final Hausman--Taylor regression may produce
near-singularity of the instrument matrix, depending on the dataset
(this was evident in the simulations). However, with modern
econometric software this does not pose a serious problem, since
redundant instruments will be dropped automatically.


\begin{figure}[p]
  \centering
\input means.tex
\caption{Mean error of estimates ($y$-axis) against number of individuals,
$N$, in sample ($x$-axis).
{\color{red} Red}: $x_1^*$ excluded from the set of instruments;
{\color{blue} blue}, $x_1^*$ included.
$T = 10$ and 5000 replications.}
\label{fig:means}
\end{figure}

\begin{figure}[p]
  \centering
\input sds.tex
\caption{Standard error of estimates ($y$-axis) against number of individuals,
$N$, in sample ($x$-axis).
{\color{red} Red}: $x_1^*$ excluded from the set of instruments;
{\color{blue} blue} $x_1^*$ included.
$T = 10$ and 5000 replications.}
\label{fig:sds}
\end{figure}

\bibliographystyle{gretl}
\bibliography{gretl}

\end{document}
