\documentclass{article}
\usepackage{doc,url,verbatim,fancyvrb}
%\usepackage{pifont}
\usepackage[authoryear]{natbib}
\usepackage[pdftex]{graphicx}
\usepackage{gretl}
\usepackage[letterpaper,body={6.3in,9.15in},top=.8in,left=1.1in]{geometry}
\usepackage{color,hyperref}

% \usepackage[a4paper,body={6.1in,9.7in},top=.8in,left=1.1in]{geometry}

\definecolor{steel}{rgb}{0.03,0.20,0.45}

\hypersetup{pdftitle={hausman\_taylor version 0.3},
            pdfauthor={Allin Cottrell},
            colorlinks=true,
            linkcolor=blue,
            urlcolor=red,
            citecolor=steel,
            bookmarksnumbered=true,
            plainpages=false
}

\begin{document}

\setlength{\parindent}{0pt}
\setlength{\parskip}{1ex}

\newcommand{\argname}[1]{\textsl{#1}}

\title{hausman\_taylor version 0.3}
\author{Allin Cottrell}
\date{October 11, 2024}
\maketitle

\section{Introduction}

This package estimates a panel-data model using the method of
\cite{hausman-taylor81}. As is well known (see the chapter titled
``Panel data'' in the \textit{Gretl User's Guide} for an extended
discussion)
%
\begin{itemize}
\item the standard fixed-effects estimator cannot handle
  time-invariant variables (since nothing remains of such variables
  after sweeping out the individual means), while
\item the random-effects estimator cannot handle (on pain of
  inconsistency) regressors that are correlated with the unobserved
  individual effects.
\end{itemize}

The Hausman--Taylor estimator bridges this gap: it permits estimation
of a model that includes both time-invariant terms and regressors that
are correlated with the individual effects. The ``price of admission''
(more formally, the condition for identification) is that there must
be at least as many time-varying exogenous regressors---``exogenous''
in the sense of being uncorrelated with the individual effects---as 
there are time-invariant regressors that are suspected of endogeneity,
that is, of being correlated with the individual effects.

\section{The model}
\label{sec:model}

Let $i=1,\dots,N$ index individuals and $t=1,\dots,T$ index time. The
model is
\begin{equation}
\label{eq:htmod}
y_{it} = \beta_0 + x_{1it}'\beta_1 + x_{2it}'\beta_2 +
 z_{1i}'\alpha_1 + z_{2i}'\alpha_2 + u_i + \epsilon_{it}
\end{equation}
where $x_1$ and $x_2$ are time-varying and $z_1$ and $z_2$ are
time-invariant. The variables $x_1$ and $z_1$ are exogenous
(uncorrelated with the individual effects, $u_i$) while $x_2$ and
$z_2$ are assumed to be so correlated. All of the regressors are
assumed to be uncorrelated with $\epsilon_{it}$.

In general, $x_{1it}$, $x_{2it}$, $z_{1i}$ and $z_{2i}$ are vectors
of length $k_1$, $k_2$, $g_1$ and $g_2$, respectively, subject to the
identification requirement $k_1 \geq g_2$.

The algorithm for the Hausman--Taylor estimator---for a balanced panel
in which the time-series length, $T$, is the same for all
individuals---is commonly given as follows:

\begin{enumerate}
\item Regress $\tilde{y} = (y_{it} - \bar{y}_i)$ on
  $\tilde{x}_1 = (x_{1it} - \bar{x}_{1i})$ and
  $\tilde{x}_2 = (x_{2it} - \bar{x}_{2i})$ to obtain initial estimates
  of $\beta_1$ and $\beta_2$. Use the residuals from this regression,
  $e_{it}$, to estimate the ``within'' error variance
  $\sigma^2_{\epsilon}$.
\item Perform an IV regression of the stacked individual means of
  $e_{it}$ on $z_1$ and $z_2$, using as instruments $z_1$ and $x_1$.
\item Use the residual variance from step 2, namely $s_2^2$, to
  estimate $\sigma^2_u$ as $s_2^2 - \hat{\sigma}^2_{\epsilon}/T$, and
  calculate the GLS coefficient
\begin{equation}
\label{eq:theta}
\theta = 1 - \left(\frac{\hat{\sigma}^2_{\epsilon}}
  {\hat{\sigma}^2_{\epsilon} + T\hat{\sigma}^2_u}\right)^{0.5}
\end{equation}
\item Finally, perform an IV regression of $y_{it}^* = (y_{it} -
  \theta\bar{y}_i)$ on $w_{it}^* = (w_{it} - \theta\bar{w}_{i})$,
  where $w_{it} = (x_{1it}, x_{2it}, z_{1i}, z_{2i})$, using as
  instruments $\tilde{x}_1$, $\tilde{x}_2$, $\bar{x}_{1}$ and
  $z_{1}$.
\end{enumerate}

In an unbalanced panel the time-series length, $T_i$, differs across
individuals. In that case step 3 above has to be modified. First, the
calculuation of $\hat{\sigma}^2_u$ uses the harmonic mean of the
$T_i$s in place of a common $T$. Second, the value of $\theta$ differs
across individuals:
%
\begin{equation}
\label{eq:theta-unbal}
\theta_i = 1 - \left(\frac{\hat{\sigma}^2_{\epsilon}}
  {\hat{\sigma}^2_{\epsilon} + T_i\hat{\sigma}^2_u}\right)^{0.5}
\end{equation}
%
Section~\ref{sec:unbal} below takes up a further issue pertaining to
the unbalanced case.

\section{The hausman\_taylor function}

The signature of this function is
%
\begin{code}
bundle hausman_taylor (series y "dependent variable", 
                       list X1 "time-varying exogenous vars",
                       list X2 "time-varying endogenous vars", 
                       list Z1 "time-invariant exogenous vars",
                       list Z2 "time-invariant endogenous vars",
                       int verbosity[0:2:1],
                       bool stata_style[1])
\end{code}

The series \texttt{y} is the dependent variable, and the four list
arguments correspond the subsets of regressors $x_1$, $x_2$, $z_1$ and
$z_2$ as shown in equation~(\ref{eq:htmod}).

The \texttt{verbosity} parameter accepts values 0, 1 or 2: a value of
0 means that nothing is printed; 1 means that the Hausman--Taylor
estimates are printed; 2 means that in addition the results of the
preliminary regressions (described in section~\ref{sec:model}) are
printed. The default value is 1.

By way of return value, this function offers a gretl bundle containing
the items shown in Table~\ref{tab:bun}.

\begin{table}[htbp]
\centering
\begin{tabular}{llp{.6\textwidth}}
  \textit{name}   & \textit{type} & \textit{description} \\[4pt]
  \texttt{yname} & string & name of the dependent variable \\
  \texttt{xnames} & string & names of regressors (space separated) \\
  \texttt{coeff} & matrix & regression coefficients \\
  \texttt{stderr} & matrix & standard errors \\
  \texttt{vcv} & matrix & variance--covariance matrix \\
  \texttt{s\_e} & scalar & square root of ``within'' variance, 
    $\hat{\sigma}^2_{\epsilon}$ \\
  \texttt{s\_u} & scalar & square root of $\hat{\sigma}^2_{u}$ \\
  \texttt{ninst} & scalar & number of instruments \\
  \texttt{nobs} & scalar & total number of obsevations used \\
  \texttt{effn} & scalar & number of units included \\
  \texttt{Tmin} & scalar & minimum $T_i$ value \\
  \texttt{Tmax} & scalar & maximum $T_i$ value \\
  \texttt{theta} & scalar & GLS coefficient, $\theta$ \\
  \texttt{Htest} & matrix & Hausman test results \\
  \texttt{Stest} & matrix & Sargan test results \\
\end{tabular}
\caption{Items in hausman\_taylor bundle}
\label{tab:bun}
\end{table}

Some comments on the bundle members follow.

\begin{itemize}

\item In the case of an unbalanced panel, the GLS coefficients
  $\theta_i$ will differ across individuals; the \texttt{theta} value
  is then the mean of the $\theta_i$s.

\item The \texttt{Htest} and \texttt{Stest} matrices, if present, are
  row vectors containing test statistic, degrees of freedom and
  $P$-value pertaining to the Hausman or Sargan test, respectively.
\end{itemize}

\section{Sample script}

The sample script for this package estimates two models of (log) wages
for 595 individuals observed annually from 1976 to 1982.  The
data---taken from the US Panel Study of Income Dynamics---were
originally analysed by \cite{cornwell88} and employed for assessing
various instrumental-variable estimators for panel models (including
the Hausman--Taylor model). These data are also analysed by
\cite{baltagi-akom90} and in chapter 7 of \cite{baltagi05}. The two
models are the original specification from Cornwell and Rupert, and a
modified specification from Baltagi.

\section{Graphical interface}

An entry-point for \textsf{hausman\_taylor} can be found under the
\textsf{Panel} sub-menu of gretl's \textsf{Model} menu: the label is
``Hausman--Taylor.'' %See Figure~\ref{fig:gui}.

% \begin{figure}[htbp]
%   \centering
%   \includegraphics[scale=0.6]{ivpanel-gui}
%   \caption{Specify arguments for ivpanel}
%   \label{fig:gui}
% \end{figure}

\section{Auxiliary printing function}

The auxiliary public function \texttt{ht\_print()} is provided to
``pretty-print'' the results contained in the bundle provided by
\texttt{hausman\_taylor()}; \texttt{ht\_print()} takes a pointer to
the bundle as its sole argument.

\section{More on the unbalanced case}
\label{sec:unbal}



\bibliographystyle{gretl}
\bibliography{gretl}

\end{document}
