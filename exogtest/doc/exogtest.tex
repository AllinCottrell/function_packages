\documentclass{article}
\usepackage{lucidabr,amsmath}
\usepackage{verbatim,fancyvrb}
\usepackage{color,gretlcode}
\usepackage[authoryear]{natbib}
\usepackage{smallerhds,url}
\usepackage[pdftex]{hyperref}

\usepackage[letterpaper,body={6.3in,9.15in},top=.8in,left=1.1in]{geometry}
%\usepackage[a4paper,body={6.1in,9.7in},top=.8in,left=1.1in]{geometry}

\definecolor{steel}{rgb}{0.03,0.20,0.45}

\hypersetup{pdftitle={exogtest 0.1},
            pdfauthor={Allin Cottrell},
            colorlinks=true,
            linkcolor=blue,
            urlcolor=red,
            citecolor=steel,
            bookmarksnumbered=true,
            plainpages=false
}


\begin{document}

\setlength{\parindent}{0pt}
\setlength{\parskip}{1ex}

\title{exogtest 0.2}
\author{Allin Cottrell}
\date{November 1, 2024}
\maketitle

\section{The difference-in-Sargan test}

This package implements the difference-in-Sargan test (sometimes
called the ``GMM distance'' test or $C$-test) for exogeneity of one or
more regressors in the context of an IV model estimated via gretl's
\texttt{tsls} command.

The overall strategy of this test is:
\begin{enumerate}
\item Estimate an IV model in which the variables in question are
  treated as endogenous; call this the restricted model.
\item Estimate a model in which the variables in question are added to
  the list of instruments; call this the unrestricted model.
\item Compute the test statistic as the Sargan test for the
  unrestricted model minus that for the restricted.
\end{enumerate}

The standard Sargan overidentification test assesses the joint null
hypothesis that an IV model is correctly specified and the instruments
are valid (asymptotically uncorrelated with the error term)---see
\citet[][section 7.8]{davidson-mackinnon93} for details. The idea
behind the difference-in-Sargan test is that one can equally well test
the marginal increase in the degree of overidentification that results
when one or more instruments are added. The logic of the test,
however, requires that the initial (restricted) model is correctly
specified with valid instruments, so if the standard Sargan test
rejects one should not place much faith in the difference-in-Sargan
test.

\section{Command-line use}

The function available for command-line and scripting use is named
\texttt{exogtest}. It takes the following arguments:

\begin{center}
\begin{tabular}{llll}
\textit{name}  & \textit{type} & \textit{comment} & \textit{default value} \\[4pt]
\textsl{model} & bundle & the initial IV model & --  \\ 
\textsl{X}     & list & the variable(s) to be tested &  -- \\
\textsl{quiet} & boolean & quiet operation? & 0 (no)
\end{tabular}
\end{center}

On successful completion \texttt{exogtest} returns a 3-vector holding
test statistic, degrees of freedom and $p$-value.

We illustrate by reference to the sample script included in the
package, which uses data on young men's wages from
\cite{griliches76}. The script begins with
%
\begin{code}
include exogtest.gfn
open griliches.gdt
# TSLS, schooling (s) treated as endogenous
series exprsq = expr^2
list xlist = const age expr exprsq s
list zlist = const expr exprsq age iq med
tsls lw xlist ; zlist
\end{code}
%
Here we estimate an overidentified IV model for the log of the wage,
with exogenous regressors age, experience and the square of experience
and potentially endogenous regressor years of schooling (\texttt{s});
the excluded instruments are IQ score and mother's education. The
Sargan test statistic for this regression is 0.028, with a $p$-value
of 0.87, so we do not reject the null of correct specification and
valid instruments. But now we'd like to see if schooling can be
treated as exogenous:
%
\begin{code}
# test the schooling variable
matrix dstest = exogtest($model, s)
print dstest
\end{code}
%$
The answer is a firm ``No'': the test statistic is 23.1 with $p$-value
$1.5\times 10^{-6}$. Note that although the second parameter to
\texttt{exogtest} is marked as a list, it's acceptable to provide as
argument a single series.

\section{Some details}

If the initial model is exactly identified then the initial Sargan
test is taken to be identically zero. Obviously, however, one has
no assurance regarding the validity of the instruments in this case.
If the initial model is overidentified and the Sargan test rejects at
the 5 percent level, a warning is printed.

If the revised (unrestricted) model has no remaining endogenous
regressors then it is estimated via OLS, otherwise by TSLS.

Each Sargan statistic is calculated as the explained sum of squares
from an auxiliary regression of the respective residuals on the full
set of instruments, divided by a measure of the error variance. To
ensure a non-negative result in finite samples the same variance
estimate is used in both cases, namely the MSE from the unrestricted
model.

Let subscript $r$ indicate the restricted model and $u$ the
unrestricted, and let ESS denote the explained sum of squares from the
associated auxiliary regression. Then the test statistic is calculated
as $S_u - S_r$, where
\[
  S_i = \frac{\mbox{ESS}_i}{\mbox{MSE}_u} \quad i=r,u
\]
The test statistic is asymptotically distributed as $\chi^2$ with
degrees of freedom equal to the number of regressors under test.

\section{Use in the gretl GUI}

This package offers to attach to the \textsf{Analysis} menu in the
model window after estimation of a model via TSLS: the menu item is
titled ``Exogeneity.'' Clicking this item brings up a dialog where you
can select a regressor to test. (At present in the GUI you are limited
to selecting a single variable.)

\section{Replication exercise}

Listing~\ref{script:stata-replic} shows an additional test case for the
package. This replicates an example shown in
\citet[][section 5]{baum07}. It is similar to the sample script (a wage
regression in which the exogeneity of a measure of education is to be
tested) but in this case the instruments for education are weak and
the difference-in-Sargan test fails to reject the null hypothesis that
education is exogenous.

\begin{script}[htbp]
  \caption{Replication of example from Baum \textit{et al}}
  \label{script:stata-replic}
Input:
\begin{scodebit}
set verbose off
include exogtest.gfn
open mroz87.gdt -q

# restrict the sample to members of the labor force
smpl LFP --dummy

# TSLS, educ treated as endogenous
series exper = AX
series expersq = AX^2
series educ = WE
series age = WA
series lwage = log(WW)
list xlist = const exper expersq educ
list zlist = const exper expersq age KL6 K618
tsls lwage xlist ; zlist
dstest = exogtest($model, educ)
\end{scodebit}
%$
Partial output from test:
\begin{scodebit}
Test for exogeneity of educ:
 chi-square(1) = 0.721676 - 0.702529 = 0.0191471 [0.8899]
\end{scodebit}
\end{script}



\bibliographystyle{gretl}
\bibliography{gretl}

\end{document}
